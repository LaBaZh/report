\documentclass{../source/zjureport}

\major{信息工程}
\name{周灿松}
\title{实验设计报告}
\stuid{3190105055}
\college{信息与电子工程学院}
\date{\today}
\lab{东4-223}
\course{数字系统设计实验}
\instructor{屈民军、唐奕}
\grades{}
\expname{实验设计报告}
\exptype{设计实验}
\partner{}

\begin{document}
    \makecover
    \makeheader

    \section{实验目的和要求}
        \subsection{实验目的}
        \begin{enumerate}
            \item 掌握音符产生的方法,了解DDS技术的应用
            \item 了解音频编解码的应用
            \item 掌握系统"自顶向下"的数字系统设计方法
        \end{enumerate}

        \subsection{实验要求}
        设计出一个能够播放四首乐曲的音乐播放器,满足如下两个要求:
        \begin{enumerate}
            \item 设置play/pause_button、next_button、reset三个按键.play/pause_button按键实现乐曲在暂停与播放之间切换,按下next_button可以播放下一首乐曲;
            \item 设置3个LED灯,LED0显示目前的播放状态(亮为播放,灭为暂停),LED1和LED2显示目前乐曲号
        \end{enumerate}

    \section{实验内容和原理}
        \subsection{主控制器模块}
            \subsubsection{设计说明}
            主控制器模块主要作用为响应用户按键信息、控制系统播放两大任务,算法流程图和书上类似,此处不表。
            \subsubsection{主控制器代码}
            \lstinputlisting[
                language = Verilog,
                caption = 主控制器模块代码
                 ]{code/mcu.v}
        
        \subsection{song_reader模块}
            \subsubsection{设计说明}
            song_reader模块任务如下:
            \begin{enumerate}
                \item 根据mcu模块的要求,选择播放乐曲。
                \item 响应note_player模块请求,从song_rom中逐个取出音符{note, duration}送给note_player模块播放
                \item 判断乐曲是否播放完毕,若播放完毕,则回复mcu模块应答信号。
            \end{enumerate}


            我们采用了书上给出的电路结构实现了上述功能,其中song_rom模块是一个只读存储器,用于存放乐曲;地址计数器计算播放的音符数,它的进位输出作为;结束判断模块利用地址计数器进位输出和从song_rom中取出的duration判断是否输出song_done信号。

            \subsubsection{结束判断模块设计}
            结束判断模块采用了老师建议的状态机方法实现:


            一、地址计数器进位输出co作为reset信号,duration单独输入,共四个状态实现,代码如下:
\begin{lstlisting}[
    caption=判断结束模块一,
    language = Verilog
    ]
//判断乐曲是否播放结束模块
module is_over (
    input [5:0] duration ,
    input reset , clk ,
    output reg done
);
    parameter RESET = 0 , PAUSE = 1 , PLAY = 2 , OUT = 3;
    reg [1:0] state=0 , nextstate;

    always @(posedge clk) begin
        if(reset) state = RESET;
        else state = nextstate;
    end

    always @(*) begin
        done = 0;
        case (state)
            RESET:begin
                done = 1;
                nextstate = PAUSE;
            end 
            PAUSE:begin
                done = 0;
                if(duration)begin
                    nextstate = PLAY;
                end
                else nextstate = PAUSE;
            end
            PLAY:begin
                done = 0;
                if(duration) nextstate = PLAY;
                else nextstate = OUT;
            end
            OUT:begin
                done = 1;
                nextstate = PAUSE;
            end
        endcase
    end
endmodule
\end{lstlisting}


            二、听了老师的讲解之后,意识到将两个信号按"duration == 0 || co"进行输入状态机会更加简单,如是将代码简化至如下:
\begin{lstlisting}[
    caption=判断结束模块二,
    language = Verilog
    ]
//判断乐曲是否播放结束模块
module is_over (
    input [5:0] duration ,
    input reset , clk ,
    output reg done
);
    parameter RESET = 0 , PAUSE = 1 , PLAY = 2 , OUT = 3;
    reg [1:0] state=0 , nextstate;

    always @(posedge clk) begin
        if(reset) state = RESET;
        else state = nextstate;
    end

    always @(*) begin
        done = 0;
        case (state)
            RESET:begin
                done = 1;
                nextstate = PAUSE;
            end 
            PAUSE:begin
                done = 0;
                if(duration)begin
                    nextstate = PLAY;
                end
                else nextstate = PAUSE;
            end
            PLAY:begin
                done = 0;
                if(duration) nextstate = PLAY;
                else nextstate = OUT;
            end
            OUT:begin
                done = 1;
                nextstate = PAUSE;
            end
        endcase
    end
endmodule
\end{lstlisting}

            \subsubsection{song_reader模块代码}
\begin{lstlisting}[
    caption=song_reader模块,
    language = Verilog
    ]
    module song_reader (
        input clk,reset,play,note_done,
        input [1:0] song,
        output reg new_note,
        output song_done,
        output [5:0] note , duration
    );
        //状态编码
        parameter RESET = 0 , NEW_NOTE = 1 , WAIT = 2 , NEXT_NOTE = 3;
        reg [1:0] state , nextstate;
    
        wire [4:0] lowaddr;//song_rom的低五位地址
        wire judge;//歌曲结束标志一
        
        //控制器时序部分
        always @(posedge clk) begin
            if(reset) state = RESET;
            else state = nextstate;
        end
    
        //控制器组合部分
        always @(*) begin
            //默认输出
            new_note = 0;
            case (state)
                RESET:begin
                    new_note = 0;
                    if(play) nextstate = NEW_NOTE;
                    else nextstate = RESET;
                end
                NEW_NOTE:begin
                    new_note = 1;
                    nextstate = WAIT;
                end  
                WAIT:begin
                    new_note = 0;
                    if(play == 0) nextstate = RESET;
                    else if(note_done) nextstate = NEXT_NOTE;
                    else nextstate = WAIT;
                end
                NEXT_NOTE:begin
                    new_note = 0;
                    nextstate = NEW_NOTE;
                end
            endcase
        end
    
        //实例化地址计数器
        counter_n #(.n(32) , .counter_bits(5)) addrCounter(.clk(clk) , .r(reset) , .en(note_done) , .q(lowaddr) , .co(judge));
        
        //实例化song_rom,取出音符
        song_rom song_rom(.clk(clk) , .dout({note,duration}) , .addr({song,lowaddr}));
    
        //实例化判断模块
        over is_over(.signal((duration==0)||co) , .reset(reset) , .clk(clk) , .done(song_done));
    
    endmodule
\end{lstlisting}

    \subsection{note_player模块}
        \subsubsection{设计说明}
        音符播放模块note_ player是本实验的核心模块,它主要任务包括以下几方面。
        \begin{enumerate}
            \item 从 song_reader模块接收需播放的音符{note, duration}
            \item 根据note值找出DDS的相位增量k
            \item 以48kHz速率从 Sine_rom取出正弦样品送给音频编解码器接口模块
            \item 当一个音符播放完成,向song_reader模块索取新的音符
        \end{enumerate}
        进一步划分模块可将note_player划分为一下各个模块:作为控制单元的控制器、记录音符标记note和DD模块相位增量k查找表关系的FreqROM、DDS模块、音符节拍计时器。

        下面给出DDS模块、节拍计时器、note_player代码

        \subsubsection{代码}
        \lstinputlisting[
            language = Verilog,
            caption = DDS模块
        ]{code/dds.v}

        \lstinputlisting[
            language = Verilog,
            caption = 节拍计时器
        ]{code/timer.v}

        \begin{lstlisting}[
            caption=note_player模块,
            language = Verilog
            ]
module note_player (
    input clk , reset , play_enable , 
    input [5:0] note_to_load,duration_to_load,
    input load_new_note,sampling_pulse,beat,
    output reg note_done,
    output [15:0] sample,
    output sample_ready);
    //状态编码
    parameter RESET = 0 , WAIT = 1 , DONE = 2 , LOAD = 3;
    reg [1:0] state , nextstate;
            
    reg timer_clear , load;
    wire timer_done;
    wire [5:0] addr;
    wire [19:0] klow;
            
    always @(posedge clk) begin
        if(reset) state = RESET;
        else state = nextstate;
    end
            
    always @(*) begin
        //默认状态
        timer_clear = 1 ; load = 0 ; note_done = 0;
                    
        case (state)
            RESET:begin
                timer_clear = 1 ; load = 0 ; note_done = 0;
                nextstate = WAIT;
            end  
            WAIT:begin
                timer_clear = 0 ; load = 0 ; note_done = 0;
                if(play_enable == 0) nextstate = RESET;
                else if(timer_done) nextstate = DONE;
                else if(load_new_note) nextstate = LOAD;
                else nextstate = WAIT;
            end
            DONE:begin
                timer_clear = 1 ; load = 0 ; note_done = 1;
                nextstate = WAIT;
            end
            LOAD:begin
                timer_clear = 1 ; load = 1 ; note_done = 0;
                nextstate = WAIT;
            end
        endcase
    end
            
    //实例化音符节拍定时器
    timer #(.counter_bits(6)) timer1(.clk(clk) , .en(beat) , .r(timer_clear) , .done(timer_done) , .n(duration_to_load));          
            
    //实例一个D寄存器
    dffre #(.n(6)) D(.d(note_to_load) , .en(load) , .r(~play_enable||reset) , .clk(clk) , .q(addr));
            
    //实例化FreqROM
    frequency_rom FreqROM(.clk(clk) , .dout(klow) , .addr(addr));
            
    //实例化DDS
    dds DDS(.K({2'b00,klow}) , .clk(clk) , .reset(~play_enable||reset) , .sampling_pulse(sampling_pulse) , .sample(sample) , .new_sample_ready(sample_ready));
                
endmodule
        \end{lstlisting}

    \subsection{同步化电路}
        \subsubsection{设计说明}
        因为音频解码接口模块和其他模块采用了不同的时钟,我们需要将二者进行同步化
        \subsubsection{代码}
        \lstinputlisting[
            language = Verilog,
            caption = 同步化电路
        ]{code/synch.v}

\section{主要仪器设备}
\begin{enumerate}
    \item 装有 Vivado和 ModelSim SE软件的计算机。
    \item Nexys Video Artix-7 FPGA多媒体音视频智能互联开发系统。
    \item 有源音箱或耳机。
\end{enumerate}

\section{操作方法和实验步骤}
\begin{enumerate}
    \item 按照书中提供的框图,将音乐播放器次顶层划分为一下几个模块:主控制器、乐曲读取、音符播放、同步化电路以及节拍基准产生器
    \item 参考实验15的资料,学习DDS技术相关知识,编写DDS模块并进行仿真
    \item 根据书上的指导,依次编写剩下模块并进行仿真验证,测试其是否符合要求
    \item 编写次顶层模块,并在其中设置参数sim方便仿真
    \item 新建Vivado工程,生成符合要求的DCM模块,将自己编写的模块以及老师提供的网表文件及接口文件加入工程。对工程进行综合、约束、实现,并下载到开发板中进行验证
\end{enumerate}

\end{document}